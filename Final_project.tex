\documentclass[11pt]{amsart}
\usepackage{amssymb, amsmath}
\usepackage{graphicx}
\newtheorem{theorem}{Theorem}[section]
\newtheorem{lemma}[theorem]{Lemma}
\newtheorem{corollary}[theorem]{Corollary}
\newtheorem{prop}[theorem]{Proposition}

\theoremstyle{definition}
\newtheorem{definition}[theorem]{Definition}
\newtheorem{example}[theorem]{Example}

\newcommand{\N}{\mathbb{N}}
\newcommand{\Z}{\mathbb{Z}}

\def\di{\partial}
\def\bs{\backslash}
\def\e{\epsilon}
\def\a{\alpha}
\def\w{\omega}
\def\b{\beta}
\def\y{\gamma}
\def\t{\tau}
\def\D{\nabla}
\def\ot{\otimes}
\def\eq{\Leftrightarrow}
\def\l{\lambda}
\def\r{\rho}
\def\d{\Delta}
\def\k{\kappa}
\def\la{\langle}
\def\ra{\rangle}
\def\p{\phi}
\def\O{\Omega}

\author{Kristian Pedersen, kristian.berseth.pedersen@gmail.com}
\begin{document}
\title[INF5620]{INF5620: Final Project}

\maketitle
\section{Introduction}
My solution of the final project in INF5620. It is somewhat rough on the edges as I have spent litteral days on bugfixing and need the remaining time to practice for the exam.
\section{a.)}
$$\r u_t = \D \cdot (\alpha (u) \D u) + f(\bar{x},t)$$
Discretizing in time and using backward euler we get:
$$\r \frac{u^k + u^{k-1}}{\d t} = \D \cdot (\alpha (u^k) \D u^k) + f(\bar{x},t^k)$$
or, setting $\k = \frac{\r}{\d t}$, $u = u^k$ and $u_p = u^{k-1}$:
$$u = u_p + \k( \D \cdot (\alpha (u) \D u) + f(\bar{x},t^k))$$
Finding the variational form of the problem:
$$\la u,v \ra = \la u_p,v \ra + \k(\la \D \cdot (\alpha (u) \D u),v \ra + \la f(\bar{x},t^k),v \ra)$$
Sorting and applying Green's Identity along with the neumann condition:
$$\la u_p,v \ra + \k \la f(\bar{x},t^k),v \ra) = \la u,v \ra + \k \la \D \cdot (\alpha (u) \D u),v \ra$$
And we find 
$$a = \la u,v \ra + \k \la \D \cdot (\alpha (u) \D u),v \ra$$
$$L = \la u_p + \k f(\bar{x},t^k),v \ra)$$

\section{b.)}
Setting $u^q = u_p$ for q = 0 and iterating over q = 1,2,3 ... we get the following systems from picard iteration:
$$- \la u_p,v \ra - \k \la f(\bar{x},t^k),v \ra)  + \la u^q,v \ra + \k \la \D \cdot (\alpha (u^{q-1}) \D u^q),v \ra = 0$$

\section{c.)}
Inserting for q = 1 and renaming $u^1$ to u we find:
$$-\la u_p,v \ra - \k \la f(\bar{x},t^k),v \ra)  + \la u,v \ra + \k \la \D \cdot (\alpha (u_p) \D u),v \ra = 0$$
The implementation of this can be found in the file dummy.py in the Solver class' Picard function.

\section{d.)}
Implemented the problem is implemented in the generate$\_$first$\_$verification(h) function in the Problem class of dummy.py and the convergance test is launched from the convergance$\_$test1 function in the Tester class. Here I would probably have changed the structure a bit and add a proper test (instead of just printings) if I hadn't been so short on time.

\section{e.)}
Implemented the problem in the generate$\_$first$\_$verification(h) function in the Problem class of Solver.py and a test running some T values is implemented in the manufactured$\_$test() function in the tester class. The error seems reasonably small, but not non-existant. It seems to scale with T, but that is not unreasonable considering the choice of error measure.
\section{f.)}
Some factors:
\begin{enumerate}
\item The error from discretization in time
\item The error from the  assumption that the picard method will converge (A guess, but I think one can find some sufficiently ugly $\a(u)$s to make it true)
\item The error from doing a finite amount (one) of picard iterations
\item Error caused while interpolating I and other functions
\item Error caused during FEniCS's solution process (Though I am unsure wether or not this is included in point 5 and 7 when FEniCS uses "exact" solvers for linear systems)
\item Errors introduced by floating point arithmetic.
\end{enumerate}
\section{g.)}
I postponed this and promptly forgot it :(
\section{h.)}
Implemented the problem in the generate$\_$gaussian(beta) function in the Problem class of Solver.py and a test running some $\beta$ values is implemented in the gaussian$\_$test() function in the tester class. I have not implemented a way to plot the solution as I am running low time and the answer is obviously wrong (for one it is invariant in beta). I have spent some time trying to locate the bug, but it is nowhere to be found.
\section{i.)}
The group finite method is based on the assumption:
$$\a(u) = \a(\Sigma_{i=1}^n \p_i u_i) \approx \a(\Sigma_{i=1}^n u_i)\p_i$$
Then for j = 1,2...n :
$$\la \a(u), \p_j \ra = \la  \a(\Sigma_{i=1}^n \p_i u_i) , \p_j \ra \approx \la \a(\Sigma_{i=1}^n u_i)\p_i , \p_j \ra = \a(\Sigma_{i=1}^n u_i) \la \p_i , \p_j \ra$$
As we're using P1 elements $\la \p_i , \p_j \ra = 0$ for $j \ne i, i+1 $ or $ i-1$. Using the standard P1 integrals from the compendium/slides we find:
$$\la \a(u), \p_j \ra  \approx \frac{h}{6}(\alpha(u_{i-1} + \alpha(u_{i}) + \alpha(u_{i+1})$$

\section{j.)}
For i = 1,2..n we can define $F_i$ as:
$$F_i = a(u,\p_i) - L(\p_j) = \int_\O (u - u_p)\p_i - \k a(u)\D u \cdot \D \p_i + f(\bar{x},t^k) dx = 0$$

Defining the Jacobian as:

$$J_{i,j} = \frac{\di F_i}{di u_j}$$

We can use the fact that L and $\p_i$ is constant with respect to $u_j$ and apply the chain rule to the second term to get:
$$J_{i,j} = \int_\O \frac{\di u}{\di u_j}\p_i + \k \frac{\di a(u)}{\di u_j}\D u \cdot \D \p_i + a(u) \D \frac{\di u }{\di u_j} \cdot \D \p_i dx$$
Inserting the galerkin expansion of u: $u = \Sigma_{i=1}^n u_i)\p_i$ and noticing that the derivation will remove all the terms but the $u_j$ term we find:
$$J_{i,j} = \int_\O \p_j \p_i + \k a'(u)\p_j \D u \cdot \D \p_i + a(u) \D \p_j \cdot \D \p_i dx$$
Where we used the core rule to simplify the derivative of a(u)

\section{k.)}
Inserting for one dimension we get:
$$J_{i,j} = \int_\O \p_j \p_i + \k a'(u)\p_j \frac{\di u}{\di x} \p_i '(x) + a(u) \p_j' \p_i' dx$$

After several attempts to get the algebra right I have given up on the rest of this exercise. I tried using the simplicity of P1 integrals and collapsing sums when using trapezodial integration. The expression still seemed extremely complex. Maybe I've made a mistake in the above calculations?

\section{l.)}
I am assuming we would find that the expression from k to be similar to a finite element approximation, as often is the case when using P1 elements and the trapezodial rule. Hard to say which without the exact expression though.
\end{document}